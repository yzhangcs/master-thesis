\chapter{攻读学位期间的成果}

\begin{itemize}
	\setlength{\itemsep}{5pt}
	      % \setlength{\parsep}{2em}

	\item \textbf{\heiti\sihao{论文}}

	      \begin{enumerate}
		      \setlength{\itemsep}{-\itemsep}  %调整间距
		            % \usecounter{numcount} % 使用计数器,初始值为0
		            % \setlength{\leftmargin}{3em} %左边界
		            % \setlength{\parsep}{-0.5ex} %段落间距
		            % \setlength{\topsep}{-10ex} %列表到上下文的垂直距离
		            % \setlength{\itemsep}{0.5ex} %条目间距
		            % \setlength{\labelsep}{0.3em} %标号和列表项之间的距离,默认0.5em
		            % \setlength{\itemindent}{1.1em} %标签缩进量
		            % \setlength{\listparindent}{0em} %段落缩进量

		      \item Wei Jiang, Zhenghua Li, \textbf{Yu Zhang}, Min Zhang. 2019.
		            \emph{HLT@SUDA at SemEval 2019 Task 1: UCCA Graph Parsing as Constituent Tree Parsing}.
		            In Proceedings of SemEval, pages 11–15, Minneapolis, Minnesota, USA.

		      \item \textbf{Yu Zhang}, Zhenghua Li, Min Zhang. 2020.
		            \emph{Efficient Second-Order TreeCRF for Neural CRF Dependency Parsing}.
		            In Proceedings of ACL, pages 3295–3305, Online. (CCF-A类会议)

		      \item \textbf{Yu Zhang}$^\ast$, Houquan Zhou$^\ast$, Min Zhang. 2020.
		            \emph{Fast and Accurate Neural CRF Constituency Parsing}.
		            In Proceedings of IJCAI, pages 4046-4053, Online. (CCF-A类会议)

		      \item Houquan Zhou$^\ast$, \textbf{Yu Zhang}$^\ast$, Zhenghua Li, Min Zhang. 2020.
		            \emph{Is POS Tagging Necessary or Even Helpful for Neural Dependency Parsing?}.
		            In Proceedings of NLPCC. (CCF-C类会议, \textbf{\textit{Best Paper Award}})

	      \end{enumerate}

	\item \textbf{\heiti\sihao{实习}}
	      \begin{enumerate}
		      \item \textsc{2020/8--2021/2}. 杭州-阿里巴巴-达摩院.
	      \end{enumerate}

	      %\item \textbf{\heiti\sihao{实习与比赛}}
	      %	\begin{enumerate}
	      %	\item \textsc{2018/5--2018/9} \quad 实习 \quad 杭州-阿里巴巴-业务平台事业部
	      %
	      %	实习期间,参与阿里巴巴商品知识图谱(藏经阁计划)的本体构建工作.
	      %	基于现有商品类目体系协助搭建智能识别系统并完成上线;
	      %	在此基础上采用关系抽取方法构建电商场景知识库,共沉淀25754条``场景-产品词''知识;
	      %	搭建基于神经网络关系抽取的完整工作流,支持天猫``618大促''和多行业场景体系建设.
	      %
	      %	\item CCF大数据与计算智能大赛(BDCI): 基于主题的文本情感分析
	      %
	      %	该赛题要求参赛者挖掘用户评论中所蕴含的主题、以及对这些主题的情感偏好,最终以<主题,情感词>序对作为输出.
	      %
	      %	经过初赛的筛选共100支队伍进入复赛,我们最终提交的结果在所有队伍中排名\textbf{5/100},获得该赛题企业单项三等奖.
	      %
	      %	\end{enumerate}

\end{itemize}
