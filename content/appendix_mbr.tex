\chapter{关于MBR解码的推导}
\label{cha:mbr-decoding}
大体上讲,解码过程是将一个模型输出的概率分布转化为对应的系统输出。
给定输入$\boldsymbol{x}$,理想情况下一个解码器将选择一个$\boldsymbol{y}$,使得$\ell(\boldsymbol{y},\boldsymbol{y}^{\ast})$最小化,其中$\ell(\boldsymbol{y},\boldsymbol{y}^{\ast})$称为代价函数,衡量$\boldsymbol{y}$与真实分配$\boldsymbol{y}^{\ast}$的差异性。
由于通常情况下解码时不存在正确答案,因此我们转而用正确答案的所有可能取值$\boldsymbol{y}^{\prime}$的平均来代替
\begin{equation}
	\label{eq:mbr}
	\boldsymbol{y}= \arg\min_{\boldsymbol{y}}\sum_{\boldsymbol{y}^{\prime}}p(\boldsymbol{y}^{\prime}\mid\boldsymbol{x})\cdot\ell(\boldsymbol{y},\boldsymbol{y}^{\prime})
\end{equation}
上述正是\textbf{最小贝叶斯风险}(Minimum Bayesian Risk, MBR)的内涵:给定输入$\boldsymbol{x}$及对应的后验分布,选择$\boldsymbol{y}$,使得\textit{期望}代价(即风险,\textit{Risk})最小化 \citep{stoyanov-eisner-2012-minimum}。

对句法分析,或其他结构化预测任务而言,通常我们采用的是最大后验概率(Maximum \textit{A Posteriori}, MAP)解码,即寻求一个使得后验概率最大化的答案\cite{jiang-2019-structured-latent}.
这是MBR解码的一个特例,而对应的代价函数$\ell^{MAP}(\cdot)$为一个简单的0-1指示函数,判断$\boldsymbol{y}$与$\boldsymbol{y}^{\prime}$是否一致
\begin{equation}
	\ell^{MAP}(\boldsymbol{y},\boldsymbol{y}^{\prime}) = \mathbbm{1}(\boldsymbol{y}\neq\boldsymbol{y}^{\prime})
\end{equation}
代入到公式~\ref{eq:mbr}得到
\begin{equation}
	\label{eq:map}
	\begin{split}
		\boldsymbol{y}^{MAP}&= \arg\min_{\boldsymbol{y}}\sum_{\boldsymbol{y}^{\prime}}p(\boldsymbol{y}^{\prime}\mid\boldsymbol{x})\cdot\ell^{MAP}(\boldsymbol{y},\boldsymbol{y}^{\prime})\\
		&=\arg\min_{\boldsymbol{y}}\sum_{\boldsymbol{y}^{\prime}}p(\boldsymbol{y}^{\prime}\mid\boldsymbol{x})\cdot\mathbbm{1}(\boldsymbol{y}\neq\boldsymbol{y}^{\prime})\\
		&=\arg\min_{\boldsymbol{y}} 1-p(\boldsymbol{y}\mid\boldsymbol{x})\\
		&=\arg\max_{\boldsymbol{y}}p(\boldsymbol{y}\mid\boldsymbol{x})
	\end{split}
\end{equation}
上式与MAP解码的选择概率最大的$\boldsymbol{y}$的目标等价。

以基于弧分解假设的一阶句法分析模型为例,式~\ref{eq:map}的MAP解码进一步写为
\begin{equation}
	\label{eq:map-dep}
	\begin{split}
		\boldsymbol{y}^{MAP}&=\arg\max_{\boldsymbol{y}}p(\boldsymbol{y}\mid\boldsymbol{x})\\
		&=\arg\max_{\boldsymbol{y}}\frac{\exp{\left(\sum_{i\rightarrow j\in \boldsymbol{y}}\mathrm{s}(i\rightarrow j)\right)}}{Z(\boldsymbol{x})\equiv \sum_{\boldsymbol{y}^{\prime}}\exp{\left(\sum_{i^{\prime}\rightarrow j\in \boldsymbol{y}^{\prime}}\mathrm{s}i^{\prime}\rightarrow j\right)}}\\
		&=\arg\max_{\boldsymbol{y}}\sum_{i\rightarrow j\in \boldsymbol{y}}\mathrm{s}(i\rightarrow j)
	\end{split}
\end{equation}
去掉共同的分母,分子为对应句法树的分值,上式的含义是获取分值最大的句法树,这个目标可以通过常见的一些解码算法(如Eisner,MST等)达到。

而对于MBR解码,我们定义代价函数为在每条弧上的指示函数,
\begin{equation}
	\ell^{MBR}(\boldsymbol{y},\boldsymbol{y}^{\prime}) = \sum_{i\rightarrow j\in\boldsymbol{y},i^{\prime}\rightarrow j\in\boldsymbol{y}^{\prime}}\mathbbm{1}(i\neq i^{\prime})
\end{equation}
代入到公式~\ref{eq:mbr}得到
\begin{equation}
	\label{eq:mbr-dep}
	\begin{split}
		\boldsymbol{y}^{MBR}&= \arg\min_{\boldsymbol{y}}\sum_{\boldsymbol{y}^{\prime}}p(\boldsymbol{y}^{\prime}\mid\boldsymbol{x})\cdot\ell^{MBR}(\boldsymbol{y},\boldsymbol{y}^{\prime})\\
		&=\arg\min_{\boldsymbol{y}}\sum_{\boldsymbol{y}^{\prime}}p(\boldsymbol{y}^{\prime}\mid\boldsymbol{x})\cdot\sum_{i\rightarrow j\in\boldsymbol{y},i^{\prime}\rightarrow j\in\boldsymbol{y}^{\prime}}\mathbbm{1}(i\neq i^{\prime})\\
		&=\arg\min_{\boldsymbol{y}}\sum_{i\rightarrow j\in\boldsymbol{y},i^{\prime}\rightarrow j\in\boldsymbol{y}^{\prime}}\sum_{\boldsymbol{y}^{\prime}}p(\boldsymbol{y}^{\prime}\mid\boldsymbol{x})\cdot\mathbbm{1}(i\neq i^{\prime})\\
		&=\arg\min_{\boldsymbol{y}}\sum_{i\rightarrow j\in\boldsymbol{y}}\sum_{\boldsymbol{y}^{\prime}:i^{\prime}\rightarrow j\notin \boldsymbol{y}^{\prime}}p(\boldsymbol{y}^{\prime}\mid\boldsymbol{x})\\
		&=\arg\min_{\boldsymbol{y}}\sum_{i\rightarrow j\in\boldsymbol{y}}\left(1-\sum_{\boldsymbol{y}^{\prime}:i^{\prime}\rightarrow j\in \boldsymbol{y}^{\prime}}p(\boldsymbol{y}^{\prime}\mid\boldsymbol{x})\right)\\
		&=\arg\max_{\boldsymbol{y}}\sum_{i\rightarrow j\in\boldsymbol{y}}\sum_{\boldsymbol{y}^{\prime}:i^{\prime}\rightarrow j\in \boldsymbol{y}^{\prime}}p(\boldsymbol{y}^{\prime}\mid\boldsymbol{x})\\
		&=\arg\max_{\boldsymbol{y}}\sum_{i\rightarrow j\in\boldsymbol{y}}p(i\rightarrow j\mid\boldsymbol{x})
	\end{split}
\end{equation}
式~\ref{eq:mbr-dep}和式~\ref{eq:map-dep}的唯一区别在于要求和的项是边缘概率而非分值。
因此,在一阶依存句法分析的场景下,MBR解码只需要给定弧的边缘概率,然后直接应用和MAP解码一样的解码算法即可 \citep{smith-smith-2007-probabilistic, smith-2011-linguistic}。

\noindent$\blacksquare$