% !Mode:: "TeX:UTF-8"

% 中英文摘要
\begin{cabstract}
  作为句子分析任务的重要中间过程之一,句法分析任务因为巨大的研究价值和重要的实用意义一直备受人们关注.
  当前最为流行的句法分析器通常结合了一个强大的编码器,并采用一个简单的局部训练目标.
  尽管简单高效,但是这种训练范式由于没有在训练时显式建模树结构,造成了训练和解码的不匹配,一定程度上限制了模型性能的提升.
  因此在本文中,为了进一步提升句法分析的性能,我们尝试将结构化学习引入到现有的神经句法分析器中,并引入高阶建模.
  我们在依存句法分析和成分句法分析的多个基准数据集上做了实验,发现带来了显著提升.
  为了解决结构化学习和高阶建模的高复杂度的问题,我们还采用了批次化学习算法,利用GPU并行计算的能力,大大降低了推断时间.
  此外,我们还尝试了变分推断作为近似推断算法代替精确推断,进一步提升了解析速度,并且达到或接近了最佳性能.

  总体而言,本文的主要研究内容包含三个方面:
  \begin{enumerate}
    \item 提出了基于树形条件随机场的高阶依存句法分析方法.
          本文在基于局部学习训练目标的Biaffine Parser的基础上,提出采用二阶树形条件随机场的拓展TreeCRF的扩展.
          本文采用了连续兄弟特征作为二阶特征,为了更有效融入二阶子树的分值,本文受双仿射结构启发,提出了一个新颖的三仿射(Triaffine)结构来打分.
          相较于传统方法需要直接进行Inside-Outside算法的运算,我们提出Inside算法进行批次化.
          受益于GPU并行计算的能力,我们将算法复杂度从$O(n^3)$降低到了$O(n^2)$,使得模型速度达到了400句/s,可以在现实模型中应用.
          我们在13个语言的27个数据集上进行了详细实验,结果表明二阶模型可以有更好的收敛表现,在全局指标表现良好,并且尤其对局部标注数据有用.
    \item 提出了在神经网络模型上应用树形条件随机场的快速精准成分句法分析方法.
          本文提出将树形条件随机场应用到成分句法分析中.
          相比于以前的方法,我们模型的优势在于可以获得树概率,对下游任务十分有帮助.
          我们对训练和解码算法进行了批次化,从而解决了其高复杂度的问题,带来了显著的效率提升.
          我们提出了简单的两阶段解析方法,和前人的一阶段解析相比结果相当,但是更加高效.
          我们在模型中提出用双仿射打分机制替换传统打分方法,我们还对编码器进行改进,发现双向LSTM中引入的诸如Dropout的策略改进可以极大提升解析的性能,不输当前最佳的基于自注意力机制的Transformer编码器的结果.
          我们在中英文三个基准数据集上做了实验,结果表明我们提出的模型结果显著超越了现有方法,并且速度达到1,000句/秒.
          我们的模型在使用BERT之后达到了现有最好的结果.
    \item 提出基于变分推断的近似推断句法分析方法.
          本文提出在依存句法和成分句法分析引入基于平均场变分推断的近似放啊代替精确推断方法.
          变分推断保留了精确推断方法使用的二阶特征,但是将算法在GPU上的复杂度从$O(n^2)$降低到了$O(n)$,大大提升了模型效率.
          对于依存句法和成分句法,我们分别引入了基于头选择目标和基于二分类目标的变分推断更新算法.
          在中英文共五个数据集上的实验结果表明,我们的方法显著超越了一阶模型达到了和二阶模型可比较的水平,并且解析速度大大超越.
          我们的变分推断方法在使用BERT之后的结果达到或接近了现有最佳结果.
  \end{enumerate}



  \vskip 21bp
    {\heiti\zihao{-4} 关键词:}
  依存句法分析,
  成分句法分析,
  树形条件随机场,
  变分推断

  \begin{flushright}
    作~~~~~~~~者:张~~~~宇

    指导老师:李正华

  \end{flushright}
\end{cabstract}


