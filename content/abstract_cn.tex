% !Mode:: "TeX:UTF-8"

% 中英文摘要
\begin{cabstract}
  句法分析任务是句子理解的重要中间过程之一.
  无论是神经网络方法还是深度学习时代以前的方法,采用基于全局概率模型的句法分析工作都非常少,主要的原因在于句法树概率推断算法的复杂度非常高.
  在本文中,我们提出将基于全局概率的方法应用到当前的句法分析模型中,相比于以前的方法,本方法的优势在于可以获得树概率,对于下游任务更加有用.
  特别地,我们将树形条件随机场(TreeCRF)方法应用到依存句法和成分句法这两个主要的句法分析任务.
  导致TreeCRF低效的主要瓶颈在于Inside-Outside算法的计算.
  为了解决这个问题,一方面,我们提出对Inside算法进行批次化,从而利用GPU的并行计算能力来加速,另一方面,我们还提出将复杂的Outside算法用高效的反向传播代替.
  在深度学习时代,模型被不断简化,无论是依存句法还是成分句法,采用局部损失(分类)都是当前的一个趋势.
  我们则在一阶TreeCRF的基础上采用了高阶拓展.
  高阶TreeCRF进一步增加了算法复杂度,为此,我们还尝试了用基于平均场变分推断的近似推断算法代替精确推断的高阶TreeCRF方法,从而增加了解析效率.
  我们在依存句法分析和成分句法分析的多个基准数据集上做了实验,发现精确推断的高阶TreeCRF带来了显著提升.
  我们提出的变分推断方法在达到或接近高阶TreeCRF方法性能的同时,大大加快了解析速度.

  总体而言,本文的主要研究内容包含三个方面:
  \begin{enumerate}
    \item 提出了基于TreeCRF的高阶依存句法分析方法.
          本文在基于局部学习训练目标的Biaffine Parser的基础上,提出二阶TreeCRF的扩展,采用邻接兄弟子树信息作为二阶特征.
          为了更有效融入二阶子树的分值,受双仿射结构启发,提出了一个新颖的三仿射(Triaffine)打分器来打分.
          二阶TreeCRF方法导致了严重效率问题,为此我们提出对Inside算法进行批次化,利用GPU并行计算的能力将算法复杂度从$O(n^3)$降低到了$O(n^2)$.
          此外,我们采用基于自动求导的反向传播取代了Outside算法,显著提升了效率,使得模型速度达到了400句每秒.
          我们在13个语言的27个数据集上进行了详细实验,结果表明二阶模型可以有更好的收敛表现,在全局指标上表现良好,并且尤其对局部标注数据有用.
    \item 提出了基于TreeCRF的高阶成分句法分析方法.
          本文提出将高阶TreeCRF应用到成分句法分析中.
          为了解决效率问题,我们应用了和依存模型中一致的批次化技术和反向传播来加速.
          此外,我们提出一个简单的两阶段解析方法,和前人的一阶段解析相比结果相当,但是更加高效.
          我们还参考了依存句法的模型架构和参数设置,提出用双仿射打分机制替换传统打分方法,发现在双向LSTM编码器中引入的诸如Dropout的策略改进可以极大提升解析的性能,不输于当前最佳的基于自注意力机制的Transformer编码器的结果.
          在中英文三个基准数据集上的实验结果表明,我们提出的模型结果显著超越了现有方法,并且速度达到1,000句每秒.
          我们的模型在使用BERT之后达到了现有最好的结果.
    \item 提出基于变分推断的高效句法分析方法.
          为了解决精确推断的TreeCRF方法高复杂度的问题,本文提出在依存句法和成分句法分析中引入基于平均场变分推断的近似方法.
          相比于高阶TreeCRF方法,变分推断将算法在GPU上的复杂度从$O(n^2)$降低到了$O(n)$,大大提升了模型效率.
          我们分别针对依存句法和成分句法设计了不同的因子图以及相应的变分推断更新方法.
          在中英文共五个数据集上的实验结果表明,我们的二阶变分推断方法显著超越了一阶模型,达到了和二阶TreeCRF模型可比较的水平,并且在依存句法和成分句法的解析速度分别达到了1,126句每秒和905句每秒.
          此外,使用BERT之后,我们的变分推断方法的结果达到或接近了现有的最佳结果.
  \end{enumerate}



  \vskip 21bp
    {\heiti\zihao{-4} 关键词:}
  依存句法分析,
  成分句法分析,
  树形条件随机场,
  变分推断

  \begin{flushright}
    作~~~~~~~~者:张~~~~宇

    指导老师:李正华

  \end{flushright}
\end{cabstract}


