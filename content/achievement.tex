\chapter{攻读学位期间的成果}

\begin{itemize}
	\setlength{\itemsep}{5pt}
	% \setlength{\parsep}{2em}

	\item \textbf{\heiti\sihao{论文}}
	      \begin{enumerate}
		      \setlength{\itemsep}{-\itemsep}  %调整间距
		      % \usecounter{numcount} % 使用计数器,初始值为0
		      % \setlength{\leftmargin}{3em} %左边界
		      % \setlength{\parsep}{-0.5ex} %段落间距
		      % \setlength{\topsep}{-10ex} %列表到上下文的垂直距离
		      % \setlength{\itemsep}{0.5ex} %条目间距
		      % \setlength{\labelsep}{0.3em} %标号和列表项之间的距离,默认0.5em
		      % \setlength{\itemindent}{1.1em} %标签缩进量
		      % \setlength{\listparindent}{0em} %段落缩进量

		      \item \emph{Efficient Second-Order TreeCRF for Neural CRF Dependency Parsing}.
		            2020. \textbf{ACL (CCF-A类会议)}. 一作.
		      \item \emph{Fast and Accurate Neural CRF Constituency Parsing}.
		            2020. \textbf{IJCAI (CCF-A类会议)}. 一作.
		      \item \emph{Is POS Tagging Necessary or Even Helpful for Neural Dependency Parsing?}.
		            2020. \textbf{NLPCC最佳论文 (CCF-C类会议)}. 共同一作.
		      \item \emph{HLT@SUDA at SemEval 2019 Task 1: UCCA Graph Parsing as Constituent Tree Parsing}.
		            2019. \textbf{SemEval}. 三作.

	      \end{enumerate}

	\item \textbf{\heiti\sihao{比赛}}
	      \begin{enumerate}
		      \item 2020语言与智能技术竞赛比赛,第六名.
		      \item 2019语义分析国际评测比赛,第一名.
	      \end{enumerate}

	\item \textbf{\heiti\sihao{实习}}
	      \begin{enumerate}
		      \item \textsc{2020/8--2021/2}. 杭州-阿里巴巴-达摩院.
	      \end{enumerate}

\end{itemize}
