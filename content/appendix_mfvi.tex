\chapter{用于结构化预测的平均场变分推断的推导}
\label{appendix:mfvi-derivation}
对于一个真正的后验分布$P(\boldsymbol{y})$,定义为
\begin{equation}\label{eq:posterior}
    P(\boldsymbol{y}) =\frac{\prod_{i=1}^{N} \phi(\boldsymbol{y}_i)}{Z:=\sum_{\boldsymbol{y}}\prod_{i=1}^{N} \phi(\boldsymbol{y}_i)}
\end{equation}
其中$Z$为partition term,我们定义未归一化的分布$\tilde{P}(\boldsymbol{y})=\prod_{i=1}^{N} \phi(\boldsymbol{y}_i)$,$\phi(\cdot)=\exp(\mathrm{s}(\cdot))$

变分推断建模一个近似分布$Q(\boldsymbol{y})$,并最小化与真实分布的KL散度(KL Divergence)
\begin{equation}
    \begin{split}
        KL(Q||P)
        &=-E_{Q(\boldsymbol{y})}\left[\log\frac{Q(\boldsymbol{y})}{P(\boldsymbol{y})}\right]\\
        &=\underbrace{E_{Q(\boldsymbol{y})}\left[\log \tilde{P}(\boldsymbol{y})\right]-E_{Q(\boldsymbol{y})}\left[\log Q(\boldsymbol{y})\right]}_{ELBo}+\log Z
    \end{split}
\end{equation}
定义
\begin{equation}
    \mathcal{L}(Q)=ELBo=E_{Q(\boldsymbol{y})}\left[\log \tilde{P}(\boldsymbol{y})\right]-E_{Q(\boldsymbol{y})}\left[\log Q(\boldsymbol{y})\right]
\end{equation}
我们将最小化KL代替为最大化ELBo
\begin{equation}
    \begin{split}
        Q^{\ast}(\boldsymbol{y}) &= \arg\max_{Q}\mathcal{L}(Q)\\
        &= \arg\max_{Q}E_{Q(\boldsymbol{y})}\left[\log \tilde{P}(\boldsymbol{y})\right]-E_{Q(\boldsymbol{y})}\left[\log Q(\boldsymbol{y}))\right]
    \end{split}
\end{equation}


Mean Field假设每个变量相互独立,有$Q(\boldsymbol{y})=\prod_i{Q_i(\boldsymbol{y}_i)}$. 因此上式可应用梯度上升法(Coordinate Ascent),每次迭代优化一个变量$Q_j(\boldsymbol{y}_j)$,保持其余变量$Q_{-j}(\boldsymbol{y}_{-j})$不变,有
\begin{equation}
    \mathcal{L}(Q)=\underbrace{E_{Q(\boldsymbol{y})}\left[\log \tilde{P}(\boldsymbol{y})\right]}_{\textcircled{1}}-\underbrace{E_{Q(\boldsymbol{y})}\left[\log Q(\boldsymbol{y})\right]}_{\textcircled{2}}
\end{equation}
其中
\begin{equation}
    \begin{split}
        \textcircled{1}&=E_{Q(\boldsymbol{y})}\left[\log \tilde{ P}(\boldsymbol{y})\right]\\
        &=\int_{\boldsymbol{y}}Q(\boldsymbol{y}) \log \tilde{P}(\boldsymbol{y})d\boldsymbol{y}\\
        &=\int_{\boldsymbol{y}}\prod_i{Q_i(\boldsymbol{y}_i)} \log \tilde{P}(\boldsymbol{y})d\boldsymbol{y}\\
        &=\int_{\boldsymbol{y}_j}Q_j(\boldsymbol{y}_j)\left(\int_{\boldsymbol{y_{-j}}}Q_{-j}(\boldsymbol{y}_{-j})\log \tilde{ P}(\boldsymbol{y}) d\boldsymbol{y}_{-j}\right) d\boldsymbol{y}_j\\
        &=\int_{\boldsymbol{y_j}}Q_j(\boldsymbol{y}_j)E_{Q_{-j}(\boldsymbol{y}_{-j})}\left[\log \tilde{ P}(\boldsymbol{y}) \right] d\boldsymbol{y}_j\\
        \textcircled{2}&=E_{Q(\boldsymbol{y})}\left[\log Q(\boldsymbol{y}))\right]\\
        &=\int_{\boldsymbol{y}}Q(\boldsymbol{y}) \sum_i\log Q_i(\boldsymbol{y}_i) d\boldsymbol{y}\\
        &=\int_{\boldsymbol{y}}\prod_i{Q_i(\boldsymbol{y}_i)} \sum_i\log Q_i(\boldsymbol{y}_i) d\boldsymbol{y}\\
        &=\sum_j\int_{\boldsymbol{y}}\prod_i{Q_i(\boldsymbol{y}_i)}\log Q_j(\boldsymbol{y}_j) d\boldsymbol{y}\\
        &=\sum_j\int_{\boldsymbol{y}_j}Q_j(\boldsymbol{y}_j)\log Q_j(\boldsymbol{y}_j) d\boldsymbol{y}_j\int_{\boldsymbol{y}_{-j}} {Q_{-j}(\boldsymbol{y}_{-j})}d\boldsymbol{y}_{-j}\\
        &=\sum_j\int_{\boldsymbol{y}_j}Q_j(\boldsymbol{y}_j)\log Q_j(\boldsymbol{y}_j) d\boldsymbol{y}_j\\
        &=\int_{\boldsymbol{y}_j}Q_j(\boldsymbol{y}_j)\log Q_j(\boldsymbol{y}_j) d\boldsymbol{y}_j + C
    \end{split}
\end{equation}
$C$代表常数项,因此
\begin{equation}
    \begin{split}
        \mathcal{L}(Q)&=\textcircled{1}-\textcircled{2}\\
        &=\int_{\boldsymbol{y_j}}Q_j(\boldsymbol{y}_j)E_{Q_{-j}(\boldsymbol{y}_{-j})}\left[\log \tilde{P}(\boldsymbol{y}) \right] d\boldsymbol{y}_j-\int_{\boldsymbol{y}_j}Q_j(\boldsymbol{y}_j)\log Q_j(\boldsymbol{y}_j) d\boldsymbol{y}_j +C\\
        &=-\int_{\boldsymbol{y_j}}Q_j(\boldsymbol{y}_j)\log\frac{Q_j(\boldsymbol{y}_j)}{\exp\left(E_{Q_{-j}(\boldsymbol{y}_{-j})}\left[\log \tilde{P}(\boldsymbol{y}) \right]\right)} d\boldsymbol{y}_j +C\\
        &=KL\left(Q_j(\boldsymbol{y}_j)||\exp\left(E_{Q_{-j}(\boldsymbol{y}_{-j})}\left[\log \tilde{P}(\boldsymbol{y}) \right]\right)\right)+C
    \end{split}
\end{equation}
因此最大化$\mathcal{L}(Q)$即最小化$KL\left(Q_j(\boldsymbol{y}_j)||\exp\left(E_{Q_{-j}(\boldsymbol{y}_{-j})}\left[\log \tilde{P}(\boldsymbol{y}) \right]\right)\right)$,相应的更新公式为
\begin{equation}
    {Q^{\ast}_j(\boldsymbol{y}_j)}=\exp\left(E_{Q_{-j}(\boldsymbol{y}_{-j})}\left[\log \tilde{P}(\boldsymbol{y}) \right]\right)
\end{equation}
对于包含二阶factor的$\boldsymbol{y}$,上式进一步写为
\begin{equation}
    \begin{split}
        {Q^{\ast}_j(\boldsymbol{y}_j)}&=\exp\left(E_{Q_{-j}(\boldsymbol{y}_{-j})}\left[\log \tilde{P}\boldsymbol{y})\right]\right)\\
        &=\exp\left(E_{Q_{-j}(\boldsymbol{y}_{-j})}\left[\sum_i s_u(\boldsymbol{y}_i)+\sum_{i,k} s_b(\boldsymbol{y}_i,\boldsymbol{y}_k) \right]\right)
    \end{split}
\end{equation}
其中,对于一阶项
\begin{equation}
    \begin{split}
        E_{Q_{-j}(\boldsymbol{y}_{-j})}\left[\sum_i s_u(\boldsymbol{y}_i)\right]
        &=\sum_{i}s_u(\boldsymbol{y}_i) \int_{\boldsymbol{y}_{-j}} Q_{-j}(\boldsymbol{y}_{-j}) d\boldsymbol{y}_{-j}\\
        &=\sum_{i}s_u(\boldsymbol{y}_i)\\
        &=s_u(\boldsymbol{y}_j)+C
    \end{split}
\end{equation}
对于二阶项
\begin{equation}
    \begin{split}
        E_{Q_{-j}(\boldsymbol{y}_{-j})}\left[\sum_{i,k} s_b(\boldsymbol{y}_i,\boldsymbol{y}_k) \right] &=\int_{\boldsymbol{y}_{-j}} Q_{-j}(\boldsymbol{y}_{-j})\left(\sum_{i,k} s_b(\boldsymbol{y}_i,\boldsymbol{y}_k) \right) d\boldsymbol{y}_{-j}\\
        &=\sum_{i,k}\int_{\boldsymbol{y}_{-j}} Q_{-j}(\boldsymbol{y}_{-j})s_b(\boldsymbol{y}_i,\boldsymbol{y}_k) d\boldsymbol{y}_{-j}\\
        &=\sum_{i\neq j}{\int_{\boldsymbol{y}_{-j}}Q_{-j}(\boldsymbol{y}_{-j}) s_b(\boldsymbol{y}_i,\boldsymbol{y}_j)  d\boldsymbol{y}_{-j}} +C\\
        &=\sum_{i\neq j}{\int_{\boldsymbol{y}_i}Q_{i}(\boldsymbol{y}_i) s_b(\boldsymbol{y}_i,\boldsymbol{y}_j)  d\boldsymbol{y}_i}{\int_{\boldsymbol{y}_{-i-j}}Q_{{-i-j}}(\boldsymbol{y}_{-i-j})   d\boldsymbol{y}_{-i-j}} +C\\
        &=\sum_{i\neq j}{\int_{\boldsymbol{y}_i}Q_{i}(\boldsymbol{y}_i) s_b(\boldsymbol{y}_i,\boldsymbol{y}_j)  d\boldsymbol{y}_i} +C
    \end{split}
\end{equation}
因此,更新公式为
\begin{equation}
    {Q^{\ast}_j(\boldsymbol{y}_j)}\propto \exp\left(s_u(\boldsymbol{y}_j) + \sum_{i\neq j}{\sum_{\boldsymbol{y}_i}Q_{i}(\boldsymbol{y}_i) s_b(\boldsymbol{y}_i,\boldsymbol{y}_j)} \right)
\end{equation}

\noindent$\blacksquare$
