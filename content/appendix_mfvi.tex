\chapter{用于结构化预测的平均场变分推断的推导}
\label{appendix:mfvi-derivation}
对于一般的结构化任务而言,我们需要得到后验分布$P(\boldsymbol{y}\mid\boldsymbol{x})$。
我们可以将输出$\boldsymbol{y}$按照对应的因子图分解为若干个因子的集合 \citep{sutton-etal-2012-crf},相应的$P(\boldsymbol{y}\mid\boldsymbol{x})$定义为\footnote{后面方便起见我们省略输入$\boldsymbol{x}$的标记}
\begin{equation}\label{eq:posterior}
	P(\boldsymbol{y}\mid\boldsymbol{x}) =\frac{\prod_{\alpha} \psi_{\alpha}(y_{\alpha})}{Z(\boldsymbol{x})\equiv\sum_{\boldsymbol{y}^{\prime}}\prod_{\alpha} \psi_{\alpha}(y^{\prime}_{\alpha})}
\end{equation}
其中$Z$为配分项,$\psi_{\alpha}(\cdot)$为因子$\alpha$的势函数,我们定义未归一化的分布$\tilde{P}(\boldsymbol{y})=\prod_{\alpha} \psi_{\alpha}(y_{\alpha})$。

变分推断建模一个近似分布$Q(\boldsymbol{y})$,并最小化其与真实分布的KL散度(Kullback–Leibler Divergence)
\begin{equation}
	\begin{split}
		Q^{\ast}(\boldsymbol{y})
		&=\arg\min_{Q} KL(Q\|P)\\
		&=\arg\min_{Q} E_{Q}\left[\log\frac{Q(\boldsymbol{y})}{P(\boldsymbol{y})}\right]\\
		&=\arg\min_{Q} E_{Q}\left[\log Q(\boldsymbol{y})\right] - E_{Q}\left[P(\boldsymbol{y})\right]\\
		&=\arg\min_{Q} \log Z-\underbrace{\left(E_{Q}\left[\log \tilde{P}(\boldsymbol{y})\right]-E_{Q}\left[\log Q(\boldsymbol{y})\right]\right)}_{ELBo}
	\end{split}
\end{equation}
去掉配分项,定义
\begin{equation}
	\mathcal{L}(Q)=ELBo=E_{Q}\left[\log \tilde{P}(\boldsymbol{y})\right]-E_{Q}\left[\log Q(\boldsymbol{y})\right]
\end{equation}
我们将最小化KL散度的目标函数代替为最大化ELBo
\begin{equation}
	\begin{split}
		Q^{\ast}(\boldsymbol{y}) &= \arg\max_{Q}\mathcal{L}(Q)\\
		&= \arg\max_{Q}E_{Q}\left[\log \tilde{P}(\boldsymbol{y})\right]-E_{Q}\left[\log Q(\boldsymbol{y}))\right]
	\end{split}
\end{equation}


Mean Field假设每个变量相互独立,即有$Q(\boldsymbol{y})=\prod_i{Q_i(y_i)}$。
因此上式可应用梯度上升法(Coordinate Ascent),每次迭代优化一个变量$Q_j(y_j)$,保持其余变量$Q_{-j}(y_{-j})$不变。
对于目标函数
\begin{equation}
	\mathcal{L}(Q)=\underbrace{E_{Q}\left[\log \tilde{P}(\boldsymbol{y})\right]}_{\tcircle{1}}-\underbrace{E_{Q}\left[\log Q(\boldsymbol{y})\right]}_{\tcircle{2}}
\end{equation}
分别有
\begin{equation}
	\begin{split}
		\tcircle{1}&=E_{Q}\left[\log \tilde{P}(\boldsymbol{y})\right]\\
		&=\int_{\boldsymbol{y}}Q(\boldsymbol{y}) \log \tilde{P}(\boldsymbol{y})d\boldsymbol{y}\\
		&=\int_{\boldsymbol{y}}\prod_i{Q_i(y_i)} \log \tilde{P}(\boldsymbol{y})d\boldsymbol{y}\\
		&=\int_{y_j}Q_j(y_j)\left(\int_{\boldsymbol{y_{-j}}}Q_{-j}(y_{-j})\log \tilde{P}(\boldsymbol{y}) dy_{-j}\right) dy_j\\
		&=\int_{y_j}Q_j(y_j)E_{Q_{-j}}\left[\log \tilde{P}(\boldsymbol{y}) \right] dy_j\\
		\tcircle{2}&=E_{Q}\left[\log Q(\boldsymbol{y}))\right]\\
		&=\int_{\boldsymbol{y}}Q(\boldsymbol{y}) \sum_i\log Q_i(y_i) d\boldsymbol{y}\\
		&=\int_{\boldsymbol{y}}\prod_i{Q_i(y_i)} \sum_i\log Q_i(y_i) d\boldsymbol{y}\\
		&=\sum_i\int_{\boldsymbol{y}}Q_i(y_i)Q_{-i}(y_{-i})\log Q_i(y_i) d\boldsymbol{y}\\
		&=\sum_i\int_{y_i}Q_i(y_i)\log Q_i(y_i) dy_i\int_{y_{-i}} {Q_{-i}(y_{-i})}dy_{-i}\\
		&=\sum_i\int_{y_i}Q_i(y_i)\log Q_i(y_i) dy_i\\
		&=\int_{y_j}Q_j(y_j)\log Q_j(y_j) dy_j + \sum_{i\neq j}\int_{y_i}Q_i(y_i)\log Q_i(y_i) dy_i  \\
		&=\int_{y_j}Q_j(y_j)\log Q_j(y_j) dy_j + C
	\end{split}
\end{equation}
$C$代表常数项,因此有
\begin{equation}
	\begin{split}
		\mathcal{L}(Q)&=\tcircle{1}-\tcircle{2}\\
		&=\int_{\boldsymbol{y_j}}Q_j(y_j)\underbrace{E_{Q_{-j}}\left[\log \tilde{P}(\boldsymbol{y}) \right]}_{\log \dot{P}_j(y_j)} dy_j-\int_{y_j}Q_j(y_j)\log Q_j(y_j) dy_j +C\\
		&=\int_{\boldsymbol{y_j}}Q_j(y_j)\log\frac{Q_j(y_j)}{\dot{P}_j(y_j)} dy_j +C\\
		&=KL\left(Q_j(y_j)\|\dot{P}_j(y_j)\right)+C
	\end{split}
\end{equation}
因此最大化$\mathcal{L}(Q)$即最小化$KL\left(Q_j(y_j)\|\dot{P}_j(y_j)\right)$。
当两个分布相等时KL散度最小,因此相应的更新公式为
\begin{equation}
	\begin{split}
		Q^{\ast}_j(y_j)
		&=\exp\left(E_{Q_{-j}}\left[\log \tilde{P}(\boldsymbol{y}) \right]\right)\\
		&=\exp\left(E_{Q_{-j}}\left[\log \prod_{\alpha}\psi_{\alpha}(y_{\alpha}) \right]\right)\\
		&=\exp\left(\int_{y_{-j}} Q_{-j}(y_{-j})\sum_{\alpha}\log \psi_{\alpha}(y_{\alpha}) dy_{-j}\right)\\
		&=\exp\left(\sum_{\alpha}\int_{y_{-j}} Q_{-j}(y_{-j})\log \psi_{\alpha}(y_{\alpha}) dy_{-j}\right)\\
		&\propto\exp\left(\sum_{\alpha:j\in \mathcal{N}(\alpha)}\int_{y_{-j}} Q_{-j}(y_{-j})\log \psi_{\alpha}(y_{\alpha}) dy_{-j}\right)\\
	\end{split}
\end{equation}
$\mathcal{N}(\alpha)$代表与因子$\alpha$连接的变量,上式中,所有与变量$j$无关的因子对应的积分之和为一个常数项。
继续推导得
\begin{equation}
	\begin{split}
		Q^{\ast}_j(y_j)
		&\propto\exp\left(\sum_{\alpha:j\in \mathcal{N}(\alpha)}\int_{y_{-j}} Q_{-j}(y_{-j})\log \psi_{\alpha}(y_{\alpha}) dy_{-j}\right)\\
		&\propto\exp\left(\sum_{\alpha:j\in \mathcal{N}(\alpha)}\int_{y_{\mathcal{N}(\alpha)-j}} Q_{\mathcal{N}(\alpha)-j}(y_{\mathcal{N}(\alpha)-j})\log \psi_{\alpha}(y_{\alpha}) dy_{\mathcal{N}(\alpha)-j}\int_{y_{-\mathcal{N}(\alpha)}} Q_{-\mathcal{N}(\alpha)}(y_{-\mathcal{N}(\alpha)}) dy_{-\mathcal{N}(\alpha)}\right)\\
		&\propto\exp\left(\sum_{\alpha:j\in \mathcal{N}(\alpha)}\int_{y_{\mathcal{N}(\alpha)-j}} Q_{\mathcal{N}(\alpha)-j}(y_{\mathcal{N}(\alpha)-j})\log \psi_{\alpha}(y_{\alpha}) dy_{\mathcal{N}(\alpha)-j}\right)\\
		&\propto\exp\left(\sum_{\alpha:j\in \mathcal{N}(\alpha)}E_{Q_{\mathcal{N}(\alpha)-j}}\log \psi_{\alpha}(y_{\alpha})\right)\\
	\end{split}
\end{equation}
因此,更新公式为
\begin{equation}
	Q^{\ast}_j(y_j)\propto\exp\left(\sum_{\alpha:j\in \mathcal{N}(\alpha)}E_{Q_{\mathcal{N}(\alpha)-j}}\log \psi_{\alpha}(y_{\alpha})\right)
\end{equation}
对于包含高阶因子的模型,我们引入记号$\alpha$和$\beta$分别代表一阶因子和高阶因子。
容易得,相应的更新公式为
\begin{equation}
	Q^{\ast}_j(y_j)\propto\psi_{\alpha}(y_{\alpha})\cdot\exp\left(\sum_{\beta:j\in \mathcal{N}(\beta)}E_{Q_{\mathcal{N}(\beta)-j}}\log \psi_{\beta}(y_{\beta})\right)
\end{equation}
\noindent$\blacksquare$
