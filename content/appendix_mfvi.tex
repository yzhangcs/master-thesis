\chapter{用于结构化预测的平均场变分推断的推导}
\label{appendix:mfvi-derivation}
对于一个真正的后验分布$P(\boldsymbol{y})$,定义为
\begin{equation}\label{eq:posterior}
  P(\boldsymbol{y}) =\frac{\prod_{i=1}^{N} \phi(y_i)}{Z:=\sum_{\boldsymbol{y}}\prod_{i=1}^{N} \phi(y_i)}
\end{equation}
其中$Z$为partition term,我们定义未归一化的分布$\tilde{P}(\boldsymbol{y})=\prod_{i=1}^{N} \phi(y_i)$,$\phi(\cdot)=\exp(\mathrm{s}(\cdot))$

变分推断建模一个近似分布$Q(\boldsymbol{y})$,并最小化与真实分布的KL散度(KL Divergence)
\begin{equation}
  \begin{split}
    KL(Q\|P)
    &=-E_{Q(\boldsymbol{y})}\left[\log\frac{Q(\boldsymbol{y})}{P(\boldsymbol{y})}\right]\\
    &=\underbrace{E_{Q(\boldsymbol{y})}\left[\log \tilde{P}(\boldsymbol{y})\right]-E_{Q(\boldsymbol{y})}\left[\log Q(\boldsymbol{y})\right]}_{ELBo}+\log Z
  \end{split}
\end{equation}
定义
\begin{equation}
  \mathcal{L}(Q)=ELBo=E_{Q(\boldsymbol{y})}\left[\log \tilde{P}(\boldsymbol{y})\right]-E_{Q(\boldsymbol{y})}\left[\log Q(\boldsymbol{y})\right]
\end{equation}
我们将最小化KL代替为最大化ELBo
\begin{equation}
  \begin{split}
    Q^{\ast}(\boldsymbol{y}) &= \arg\max_{Q}\mathcal{L}(Q)\\
    &= \arg\max_{Q}E_{Q(\boldsymbol{y})}\left[\log \tilde{P}(\boldsymbol{y})\right]-E_{Q(\boldsymbol{y})}\left[\log Q(\boldsymbol{y}))\right]
  \end{split}
\end{equation}


Mean Field假设每个变量相互独立,有$Q(\boldsymbol{y})=\prod_i{Q_i(y_i)}$. 因此上式可应用梯度上升法(Coordinate Ascent),每次迭代优化一个变量$Q_j(y_j)$,保持其余变量$Q_{-j}(y_{-j})$不变,有
\begin{equation}
  \mathcal{L}(Q)=\underbrace{E_{Q(\boldsymbol{y})}\left[\log \tilde{P}(\boldsymbol{y})\right]}_{\textcircled{1}}-\underbrace{E_{Q(\boldsymbol{y})}\left[\log Q(\boldsymbol{y})\right]}_{\textcircled{2}}
\end{equation}
其中
\begin{equation}
  \begin{split}
    \textcircled{1}&=E_{Q(\boldsymbol{y})}\left[\log \tilde{ P}(\boldsymbol{y})\right]\\
    &=\int_{\boldsymbol{y}}Q(\boldsymbol{y}) \log \tilde{P}(\boldsymbol{y})d\boldsymbol{y}\\
    &=\int_{\boldsymbol{y}}\prod_i{Q_i(y_i)} \log \tilde{P}(\boldsymbol{y})d\boldsymbol{y}\\
    &=\int_{y_j}Q_j(y_j)\left(\int_{\boldsymbol{y_{-j}}}Q_{-j}(y_{-j})\log \tilde{ P}(\boldsymbol{y}) dy_{-j}\right) dy_j\\
    &=\int_{\boldsymbol{y_j}}Q_j(y_j)E_{Q_{-j}(y_{-j})}\left[\log \tilde{ P}(\boldsymbol{y}) \right] dy_j\\
    \textcircled{2}&=E_{Q(\boldsymbol{y})}\left[\log Q(\boldsymbol{y}))\right]\\
    &=\int_{\boldsymbol{y}}Q(\boldsymbol{y}) \sum_i\log Q_i(y_i) d\boldsymbol{y}\\
    &=\int_{\boldsymbol{y}}\prod_i{Q_i(y_i)} \sum_i\log Q_i(y_i) d\boldsymbol{y}\\
    &=\sum_j\int_{\boldsymbol{y}}\prod_i{Q_i(y_i)}\log Q_j(y_j) d\boldsymbol{y}\\
    &=\sum_j\int_{y_j}Q_j(y_j)\log Q_j(y_j) dy_j\int_{y_{-j}} {Q_{-j}(y_{-j})}dy_{-j}\\
    &=\sum_j\int_{y_j}Q_j(y_j)\log Q_j(y_j) dy_j\\
    &=\int_{y_j}Q_j(y_j)\log Q_j(y_j) dy_j + C
  \end{split}
\end{equation}
$C$代表常数项,因此
\begin{equation}
  \begin{split}
    \mathcal{L}(Q)&=\textcircled{1}-\textcircled{2}\\
    &=\int_{\boldsymbol{y_j}}Q_j(y_j)E_{Q_{-j}(y_{-j})}\left[\log \tilde{P}(\boldsymbol{y}) \right] dy_j-\int_{y_j}Q_j(y_j)\log Q_j(y_j) dy_j +C\\
    &=-\int_{\boldsymbol{y_j}}Q_j(y_j)\log\frac{Q_j(y_j)}{\exp\left(E_{Q_{-j}(y_{-j})}\left[\log \tilde{P}(\boldsymbol{y}) \right]\right)} dy_j +C\\
    &=KL\left(Q_j(y_j)\|\exp\left(E_{Q_{-j}(y_{-j})}\left[\log \tilde{P}(\boldsymbol{y}) \right]\right)\right)+C
  \end{split}
\end{equation}
因此最大化$\mathcal{L}(Q)$即最小化$KL\left(Q_j(y_j)\|\exp\left(E_{Q_{-j}(y_{-j})}\left[\log \tilde{P}(\boldsymbol{y}) \right]\right)\right)$,相应的更新公式为
\begin{equation}
  {Q^{\ast}_j(y_j)}=\exp\left(E_{Q_{-j}(y_{-j})}\left[\log \tilde{P}(\boldsymbol{y}) \right]\right)
\end{equation}
对于包含二阶factor的$\boldsymbol{y}$,上式进一步写为
\begin{equation}
  \begin{split}
    {Q^{\ast}_j(y_j)}&=\exp\left(E_{Q_{-j}(y_{-j})}\left[\log \tilde{P}\boldsymbol{y})\right]\right)\\
    &=\exp\left(E_{Q_{-j}(y_{-j})}\left[\sum_i s(y_i)+\sum_{i,k} s(y_i,y_k) \right]\right)
  \end{split}
\end{equation}
其中,对于一阶项
\begin{equation}
  \begin{split}
    E_{Q_{-j}(y_{-j})}\left[\sum_i s(y_i)\right]
    &=\sum_{i}s(y_i) \int_{y_{-j}} Q_{-j}(y_{-j}) dy_{-j}\\
    &=\sum_{i}s(y_i)\\
    &=s(y_j)+C
  \end{split}
\end{equation}
对于二阶项
\begin{equation}
  \begin{split}
    E_{Q_{-j}(y_{-j})}\left[\sum_{i,k} s(y_i,y_k) \right] &=\int_{y_{-j}} Q_{-j}(y_{-j})\left(\sum_{i,k} s(y_i,y_k) \right) dy_{-j}\\
    &=\sum_{i,k}\int_{y_{-j}} Q_{-j}(y_{-j})s(y_i,y_k) dy_{-j}\\
    &=\sum_{i\neq j}{\int_{y_{-j}}Q_{-j}(y_{-j}) s(y_i,y_j)  dy_{-j}} +C\\
    &=\sum_{i\neq j}{\int_{y_i}Q_{i}(y_i) s(y_i,y_j)  dy_i}{\int_{y_{-i-j}}Q_{{-i-j}}(y_{-i-j})   dy_{-i-j}} +C\\
    &=\sum_{i\neq j}{\int_{y_i}Q_{i}(y_i) s(y_i,y_j)  dy_i} +C
  \end{split}
\end{equation}
因此,更新公式为
\begin{equation}
  {Q^{\ast}_j(y_j)}\propto \exp\left(s(y_j) + \sum_{i\neq j}{\sum_{y_i}Q_{i}(y_i) s(y_i,y_j)} \right)
\end{equation}

\noindent$\blacksquare$
