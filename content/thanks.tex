\chapter{致谢}

%养天地之正气,法古今之完人. 转眼间宝贵的研究生生活即将结束,在这段成长最迅速、收获最丰富的阶段,自己学会了很多也成熟了许多. 期间,有太多的人需要感谢.
%
%首先,感谢我的导师张民教授. 感谢张老师让我成为团队中的一员,也很感谢张老师为我提供了良好的学习和科研环境. 在三年时间里,张老师的悉心指导和培养使我极大地拓宽了自己的视野和思维. 同时,张老师渊博的知识、敏锐的思维以及平易近人的性格,深深地影响着我. 在此对张老师致以我最衷心的感谢.
%
%感谢我的指导老师李正华老师. 感谢李老师在我研究生过程的每一个阶段给予的悉心指导和帮助. 李老师不仅在学术上给予极大的支持,在生活上的帮助也使得自己获益匪浅,同时,李老师饱满的工作热情和活跃的思维深深的影响了我,这三年的经历将成为我一生的财富.
%
%还要感谢周国栋、朱巧明、陈文亮、熊德意、段湘煜、姚建民、洪宇、贡正仙、孔芳、王红玲、钱龙华、李寿山、李军辉和李培峰老师等苏州大学自然语言处理实验室的所有老师,共同营造了非常好的科研环境,他们对科研的热情深深感染了我,促使我不断努力.
%
%感谢王星、丁扬、杨明明、马春平和王超超师兄,秦彦霞师姐,对我的学习给予了极大帮助;感谢我的同届伙伴龚慧敏、陈志鹏、奚浏、李方圆和唐海庆,我们互帮互助;感谢凡子威、陈伟、夏庆荣、朱运和江心舟师弟,张月、龚晨、孙佳伟和郭丽娟师妹等,陪伴我度过难忘的三年时光.
%
%感谢我的父母和家人. 在生活上和精神上给以我巨大的照顾和安慰,能让我全身心投入到学习研究工作中,同时使我充满信心,面对更多的挑战.
%
%最后,我要向百忙之中抽时间对本文进行审阅的各位老师表示衷心的感谢,谢谢各位老师提出的宝贵的修改意见.
养天地之正气,法古今之完人.
转眼间七年的求学生涯即将结束,美丽的苏州大学伴我度过了本科和硕士研究生这段宝贵的时光.
期间,我习得了知识,也收获了经历与成熟,从一开始的迷茫焦躁逐渐变得坚定从容.
回首往昔,有太多需要感谢的人.

%我的研究生生活就像这篇文章一样已经接近尾声,但这三年是我整个求学生涯中最开心、充实,也是成长最快的阶段.
%现在想来过去作出的抉择、面临的困境和收获的喜悦都是无比珍贵的回忆,所以我需要向我遇到的所有人表达由衷的感谢之情.

首先,感谢我的导师李正华副教授. 感谢李老师从学业到生活上的帮助. 学业上悉心指导,始终指引着我前进;生活上以一个朋友的身份分享自己的经历,使我受益匪浅.
同时,李老师饱满的工作热情和严谨的科研态度深深地影响了我. 在此对李老师致以最衷心的感谢和祝福.

%在这三年里,陈老师提供了良好的学习和科研环境,使我扩宽了视野和思维,更加有信心迎接今后的挑战. 陈老师对于问题的理解能力和扎实的学术功底让我十分佩服,每次课题讨论的时候,陈老师总能准确把握问题的关键所在,同时尊重我的思考方式和想法,让我受益匪浅. 陈老师是我科研和学习上的领路人,也是我今后工作和生活中的榜样.
感谢尊敬的张民教授. 感谢张老师以高标准要求团队,营造了浓厚的科研氛围. 同时,张老师渊博的知识和平易近人的性格使我获益良多.
感谢尊敬的陈文亮教授. 陈老师不仅乐于组织学术分享会而且热情地帮助组内的每一位学生.

还要感谢周国栋、朱巧明、段湘煜、洪宇、李寿山、李军辉和李培峰等苏州大学自然语言处理实验室的所有老师,大家严谨的治学态度和对科研的热情深深感染了我,激励着我不断向前.

感谢夏庆荣师兄和张月师姐每每在我遇到困难时对我伸出援助之手. 感谢郁俊杰、凡子威、陈伟、朱运、杨耀晟和何正求师兄,李英、龚晨、孙佳伟和郭丽娟师姐对我学习和生活中的帮助;感谢同届同学江心舟、朱宗奎、杨一帆、黄德朋、彭雪、周俊佐、张栋和郁圣卫,我们一起努力,共同进步. 感谢张宇、蒋炜、陆凯华、吴琨、周厚全和沈嘉钰师弟,刘亚慧、周明月和杨浩苹师妹,大家和谐相处,一起度过美好的求学时光.

感谢我的父母和家人,他们总是在任何时候给予我最无私的帮助. 他们是我心灵上的寄托,一次次驱散我心中的疲惫和阴霾,激励我不断面对挑战.

最后,我要向百忙之中抽时间对本文进行审阅的各位老师表示衷心的感谢,谢谢各位老师提出的宝贵的修改意见.

