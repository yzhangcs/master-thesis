\chapter{致谢}

养天地正气,法古今完人.
从本科到硕士,转眼间在美丽的苏州大学校园内度过了七年的求学时光.
在这段不算短的人生旅途中,无论是学识上还是生活阅历上我都成长良多.

首先,我要感谢我的导师李正华老师.
李老师永远以饱满的热情和专注的态度面对工作和生活,永远是我以后希望成为一个独立研究者的榜样.

感谢尊敬的张民老师,张老师以高标准要求每一个学生,营造了组内专一浓厚的科研氛围.
此外,张老师敏锐的思维、渊博的知识、平易近人的风格、深深的影响了我,平时的相处让我获益良多.
感谢陈文亮老师,陈老师开朗热情,在学业上也给予了我很多指导.

同样感谢周国栋、朱巧明、李寿山、洪宇和李军辉等苏州大学自然语言处理实验室的所有老师,各位老师严谨的治学态度和进取的专业精神是我的榜样.

感谢同组的夏庆荣师兄、龚晨师姐、李英师姐和张月师姐,各位师兄师姐总是十分热心的解决我生活和研究上遇到的困难.
感谢章波、黄德朋、江心舟师兄,彭雪师姐,在我还是萌新的时候对我的帮助,以及平时对我的关照.
感谢同届的蒋炜、陆凯华、吴锟、刘亚慧同学,大家在一起互相帮助,互相进步.
感谢周厚全、沈嘉钰、李嘉诚、侯洋、李帅克、周仕林、刘泽洋、李扬师弟,还有周明月、杨浩萍师妹,十分珍惜与大家相处的美好时光.
此外,还要感谢实习期间相处的蒋勇师兄以及王新宇、胡泽川、蔡炯和马欣尹同学,特别是感谢蒋勇师兄在我实习期间对我的关照,以及在课题研究上的悉心帮助和指导.

感谢我的父母还有家人们,你们总是我心灵上的港湾和寄托,无论何时都能给我最无私的帮助.

最后,我还要感谢各位评审老师,感谢各位老师们在百忙之中抽取时间对本文进行评审,并提出宝贵的修改意见.

