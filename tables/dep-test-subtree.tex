\begin{table}[tb!]
    \setlength{\tabcolsep}{5pt}
    \centering
    \caption{test数据上子树和完全树的结果.}
    \begin{tabular}{llccccc}
        \toprule
                                 &                & \multicolumn{3}{c}{SIB} & \multirow{2}{*}{UCM} & \multirow{2}{*}{LCM}                                   \\
                                 &                & P                       & R                    & F                                                      \\[2pt]
        \hline
        \\[-15pt]
        \multirow{3}{*}{PTB}     & \textsc{Loc}   & 91.16                   & 90.80                & 90.98                & 61.59          & 50.66          \\
                                 & \textsc{Crf}   & 91.24                   & 90.92                & 91.08                & 61.92          & 50.33          \\
                                 & \textsc{Crf2o} & \textbf{91.56}          & \textbf{91.11}       & \textbf{91.33}       & \textbf{63.08} & \textbf{50.99} \\[2pt]
        \hline
        \\[-15pt]
        \multirow{3}{*}{CoNLL09} & \textsc{Loc}   & 79.20                   & 79.02                & 79.11                & 40.10          & 28.91          \\
                                 & \textsc{Crf}   & 79.17                   & 79.55                & 79.36                & 40.61          & 29.38          \\
                                 & \textsc{Crf2o} & \textbf{81.00}          & \textbf{80.63}       & \textbf{80.82}       & \textbf{42.53} & \textbf{30.09} \\
        \bottomrule
    \end{tabular}
    \label{table:dev-test-subtree}
\end{table}